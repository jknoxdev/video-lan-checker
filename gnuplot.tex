% Created 2023-06-08 Thu 01:08
% Intended LaTeX compiler: pdflatex
\documentclass[11pt]{article}
\usepackage[utf8]{inputenc}
\usepackage[T1]{fontenc}
\usepackage{graphicx}
\usepackage{longtable}
\usepackage{wrapfig}
\usepackage{rotating}
\usepackage[normalem]{ulem}
\usepackage{amsmath}
\usepackage{amssymb}
\usepackage{capt-of}
\usepackage{hyperref}
\author{Eric Schulte}
\date{\today}
\title{Plotting tables in Org-Mode using org-plot}
\hypersetup{
 pdfauthor={Eric Schulte},
 pdftitle={Plotting tables in Org-Mode using org-plot},
 pdfkeywords={},
 pdfsubject={},
 pdfcreator={Emacs 28.2 (Org mode 9.5.5)}, 
 pdflang={English}}
\begin{document}

\maketitle
\tableofcontents

\section*{plot tests}
\label{sec:org1c8d14f}
\begin{center}
\begin{tabular}{lrr}
Sede & Max cites & H-index\\
\hline
Chile & 257.72 & 21.39\\
Leeds & 165.77 & 19.68\\
Sao Paolo & 71.00 & 11.50\\
Stockholm & 134.19 & 14.33\\
Morelia & 257.56 & 17.67\\
\end{tabular}
\end{center}

\section*{radar}
\label{sec:org873160c}
\begin{center}
\begin{tabular}{lrrrrrrr}
Format & Fine-grained-control & Initial Effort & Syntax simplicity & Editor Support & Integrations & Ease-of-referencing & Versatility\\
\hline
Word & 2 & 4 & 4 & 2 & 3 & 2 & 2\\
\LaTeX{} & 4 & 1 & 1 & 3 & 2 & 4 & 3\\
Org Mode & 4 & 2 & 3.5 & 1 & 4 & 4 & 4\\
Markdown & 1 & 3 & 3 & 4 & 3 & 3 & 1\\
Markdown + Pandoc & 2.5 & 2.5 & 2.5 & 3 & 3 & 3 & 2\\
\end{tabular}
\end{center}

\section*{org integration:}
\label{sec:org880d625}

\begin{table}[htbp]
\label{tab:orge79db4b}
\centering
\begin{tabular}{lrr}
Ben & 9.2 & 9.9\\
Tim & 6.7 & 7.7\\
Tom & 7.5 & 6.7\\
Dean & 8.0 & 7.0\\
\end{tabular}
\end{table}

\begin{center}
\includesvg[width=.9\linewidth]{./img/grades}
\end{center}

\section*{from the source}
\label{sec:org83dc442}

\section*{Introduction}
\label{sec:orgbcc81c8}

This tutorial provides instructions for installing and using org-plot
as well as a complete overview of the org-plot options and
demonstrations of some of the many types of graphs which can be
generated automatically from tables in your org-mode files using
org-plot.

While graphs will be included in the html version of this tutorial, if
you would like to play along at home you can download the original org
file here \href{https://git.sr.ht/\~bzg/worg/tree/master/item/org-tutorials/org-plot.org}{org-plot.org}.

\section*{Getting Set up}
\label{sec:org5e37d6f}

\subsection*{Requirements}
\label{sec:org4ab1f96}

Org-plot uses Gnuplot as well as the Emacs Gnuplot-mode to power its
graphing.  To download and install these two requirements see the
following


\begin{description}
\item[{Gnuplot}] \url{http://www.gnuplot.info/}
\item[{Gnuplot-Mode}] \url{http://cars9.uchicago.edu/\~ravel/software/gnuplot-mode.html}
\end{description}

\subsection*{Installation}
\label{sec:org158604d}
First we need to make sure that org-plot is on your system and
available to emacs.  If you have a recent version of org-mode (version
6.07 or later) then org-plot is already included.  Otherwise you can
grab the latest org-plot.el from \href{http://github.com/eschulte/org-plot/tree/master}{github/eschulte/org-plot}.  Once you
have org-plot loaded it may be useful to bind the main plotting
command \texttt{org-plot/gnuplot} to a key chord, I use \texttt{C-M-g} for the
mnemonic "graph" which can be done by executing the following elisp
snippet.  This command will be the only org-plot command needed for
nthe remainder of this tutorial.
n
\url{(local-set-key "\\M-\\C-g" 'org-plot/gnuplot)}

Alright, we're now ready to start generating some graphs.

\section*{Examples}
\label{sec:org19ce8f2}
\subsection*{2d plots (lines and histograms)}
\label{sec:orgcf83c4b}

First, to plot the following table use the key sequence defined above
\texttt{C-M-g}.  This will call \texttt{org-plot/gnuplot} which finds and plots the
nearest table.  The options specified in any \texttt{\#+PLOT} lines above the
table are read and applied to the plot.  Notice that the second
\texttt{\#+PLOT:} line specifies labels for each column, if this line is
removed the labels will default to the column headers in the table,
try removing this line and re-plotting.

\begin{center}
\includegraphics[width=.9\linewidth]{../images/org-plot/example-1.png}
\end{center}

For a complete list of all of the options and their meanings see the
\ref{org7dee608} section at the end of this file.  For more information on
gnuplot options see \href{http://gnuplot.sourceforge.net/documentation.html}{the gnuplot documentation}, nearly all gnuplot
options should be accessible through org-plot.

\begin{table}[htbp]
\label{tab:org5554e09}
\centering
\begin{tabular}{rrr}
independent var & first dependent var & second dependent var\\
\hline
0.1 & 0.425 & 0.375\\
0.2 & 0.3125 & 0.3375\\
0.3 & 0.24999993 & 0.28333338\\
0.4 & 0.275 & 0.28125\\
0.5 & 0.26 & 0.27\\
0.6 & 0.25833338 & 0.24999993\\
0.7 & 0.24642845 & 0.23928553\\
0.8 & 0.23125 & 0.2375\\
0.9 & 0.23333323 & 0.2333332\\
1 & 0.2225 & 0.22\\
1.1 & 0.20909075 & 0.22272708\\
1.2 & 0.19999998 & 0.21458333\\
1.3 & 0.19615368 & 0.21730748\\
1.4 & 0.18571433 & 0.21071435\\
1.5 & 0.19000008 & 0.2150001\\
1.6 & 0.1828125 & 0.2046875\\
1.7 & 0.18088253 & 0.1985296\\
1.8 & 0.17916675 & 0.18888898\\
1.9 & 0.19342103 & 0.21315783\\
2 & 0.19 & 0.21625\\
2.1 & 0.18214268 & 0.20714265\\
2.2 & 0.17727275 & 0.2022727\\
2.3 & 0.1739131 & 0.1989131\\
2.4 & 0.16770833 & 0.1916667\\
2.5 & 0.164 & 0.188\\
2.6 & 0.15769238 & 0.18076923\\
2.7 & 0.1592591 & 0.1888887\\
2.8 & 0.1598214 & 0.18928565\\
2.9 & 0.15603453 & 0.1844828\\
\end{tabular}
\end{table}


Org-plot can also produce histograms from 2d data, plot the following
table.  Notice that the column specified as \texttt{ind} contains textual
non-numeric data, when this is the case org-plot will use the data as
labels for the x-axis using the gnuplot \texttt{xticlabels()} function.

\begin{center}
\includegraphics[width=.9\linewidth]{../images/org-plot/example-2.png}
\end{center}

\begin{center}
\begin{tabular}{lrr}
Sede & Max cites & H-index\\
\hline
Chile & 257.72 & 21.39\\
Leeds & 165.77 & 19.68\\
São Paolo & 71.00 & 11.50\\
Stockholm & 134.19 & 14.33\\
Morelia & 257.56 & 17.67\\
\end{tabular}
\end{center}


For another example of plotting histograms instead of lines, change
the following options on the first table on this page, and replot
\begin{enumerate}
\item remove the \texttt{ind:1} option
\item replace the \texttt{with:lines} option with \texttt{with:histograms}
\end{enumerate}

\begin{center}
\includegraphics[width=.9\linewidth]{../images/org-plot/example-3.png}
\end{center}

\subsection*{3d grid plots}
\label{sec:org6d7a0c0}

There are also some functions for plotting 3d or grid data.  To see an
example of a grid plot call org-plot/gnuplot \texttt{C-M-g} which will plot
the following table as a grid.

\begin{center}
\includegraphics[width=.9\linewidth]{../images/org-plot/example-4.png}
\end{center}

To see the effect of \texttt{map} try setting it to \texttt{t}, and then
re-plotting.

\begin{center}
\includegraphics[width=.9\linewidth]{../images/org-plot/example-5.png}
\end{center}

\begin{center}
\begin{tabular}{rrrrrrrrrrrrrrrrr}
0 & 0 & 0 & 0 & 0 & 0 & 0 & 0 & 0 & 0 & 0 & 0 & 0 & 0 & 0 & 0 & 0\\
0 & 0 & 0 & 0 & 0 & 0 & 0 & 0 & 0 & 0 & 0 & 0 & 0 & 0 & 0 & 0 & 0\\
0 & 0 & 0 & 0 & 0 & 0 & 0 & 0 & 0 & 0 & 0 & 0 & 0 & 0 & 0 & 0 & 0\\
0 & 0 & 1 & 1 & 0 & 0 & 1 & 0 & 0 & 0 & 0 & 0 & 1 & 1 & 1 & 0 & 0\\
0 & 1 & 0 & 0 & 1 & 0 & 1 & 0 & 0 & 0 & 0 & 1 & 0 & 0 & 0 & 1 & 0\\
0 & 1 & 0 & 0 & 1 & 0 & 1 & 0 & 0 & 0 & 0 & 1 & 0 & 0 & 0 & 1 & 0\\
0 & 1 & 0 & 0 & 1 & 0 & 1 & 0 & 0 & 0 & 0 & 1 & 0 & 1 & 1 & 1 & 0\\
0 & 1 & 0 & 0 & 1 & 0 & 1 & 0 & 0 & 0 & 0 & 1 & 0 & 0 & 0 & 0 & 0\\
0 & 1 & 0 & 0 & 1 & 0 & 1 & 0 & 0 & 0 & 0 & 1 & 0 & 0 & 0 & 0 & 0\\
0 & 1 & 0 & 0 & 1 & 0 & 1 & 0 & 0 & 0 & 0 & 1 & 0 & 0 & 0 & 1 & 0\\
0 & 1 & 0 & 0 & 1 & 0 & 1 & 1 & 0 & 1 & 0 & 1 & 0 & 0 & 0 & 1 & 0\\
0 & 0 & 1 & 1 & 0 & 0 & 1 & 0 & 1 & 1 & 0 & 0 & 1 & 1 & 1 & 0 & 0\\
0 & 0 & 0 & 0 & 0 & 0 & 0 & 0 & 0 & 0 & 0 & 0 & 0 & 0 & 0 & 0 & 0\\
0 & 0 & 0 & 0 & 0 & 0 & 0 & 0 & 0 & 0 & 0 & 0 & 0 & 0 & 0 & 0 & 0\\
\end{tabular}
\end{center}

Plotting grids also respects the independent variable (\texttt{ind:}) option,
and uses the values of the independent row to label the resulting
graph.  The following example borrows a short description of org-mode
from Bernt Hansen on the mailing list (a more practical usage would
label every single row with something informative).

\begin{center}
\includegraphics[width=.9\linewidth]{../images/org-plot/example-6.png}
\end{center}

\begin{center}
\begin{tabular}{lrrrrrrrrrrrrrrrrr}
text & 0 & 0 & 0 & 0 & 0 & 0 & 0 & 0 & 0 & 0 & 0 & 0 & 0 & 0 & 0 & 0 & 0\\
 & 0 & 0 & 0 & 0 & 0 & 0 & 0 & 0 & 0 & 0 & 0 & 0 & 0 & 0 & 0 & 0 & 0\\
plain & 0 & 0 & 1 & 1 & 0 & 0 & 1 & 0 & 0 & 0 & 0 & 0 & 1 & 1 & 1 & 0 & 0\\
 & 0 & 1 & 0 & 0 & 1 & 0 & 1 & 0 & 0 & 0 & 0 & 1 & 0 & 0 & 0 & 1 & 0\\
in & 0 & 1 & 0 & 0 & 1 & 0 & 1 & 0 & 0 & 0 & 0 & 1 & 0 & 0 & 0 & 1 & 0\\
 & 0 & 1 & 0 & 0 & 1 & 0 & 1 & 0 & 0 & 0 & 0 & 1 & 0 & 1 & 1 & 1 & 0\\
everything & 0 & 1 & 0 & 0 & 1 & 0 & 1 & 0 & 0 & 0 & 0 & 1 & 0 & 0 & 0 & 0 & 0\\
 & 0 & 1 & 0 & 0 & 1 & 0 & 1 & 0 & 0 & 0 & 0 & 1 & 0 & 0 & 0 & 0 & 0\\
track & 0 & 1 & 0 & 0 & 1 & 0 & 1 & 0 & 0 & 0 & 0 & 1 & 0 & 0 & 0 & 1 & 0\\
 & 0 & 1 & 0 & 0 & 1 & 0 & 1 & 1 & 0 & 1 & 0 & 1 & 0 & 0 & 0 & 1 & 0\\
and & 0 & 0 & 1 & 1 & 0 & 0 & 1 & 0 & 1 & 1 & 0 & 0 & 1 & 1 & 1 & 0 & 0\\
 & 0 & 0 & 0 & 0 & 0 & 0 & 0 & 0 & 0 & 0 & 0 & 0 & 0 & 0 & 0 & 0 & 0\\
Organize & 0 & 0 & 0 & 0 & 0 & 0 & 0 & 0 & 0 & 0 & 0 & 0 & 0 & 0 & 0 & 0 & 0\\
\end{tabular}
\end{center}

\subsection*{3d plots}
\label{sec:org690cae9}

Finally the last type of graphing currently supported is 3d graphs of
data in a table.  This will probably require some more knowledge of
gnuplot to make full use of the many options available.

\begin{center}
\includegraphics[width=.9\linewidth]{../images/org-plot/example-7.png}
\end{center}

For some simple demonstrations try the following graph using some
different \texttt{with:} options \texttt{with:points}, \texttt{with:lines}, and
\texttt{with:pm3d}.

\begin{center}
\includegraphics[width=.9\linewidth]{../images/org-plot/example-8.png}
\end{center}

\begin{center}
\begin{tabular}{rrrrrrr}
0 & 0 & 0 & 0 & 0 & 0 & 0\\
0 & 2 & 2 & 2 & 2 & 2 & 0\\
0 & 2 & 3 & 3 & 3 & 2 & 0\\
0 & 2 & 3 & 4 & 3 & 2 & 0\\
0 & 2 & 3 & 3 & 3 & 2 & 0\\
0 & 2 & 2 & 2 & 2 & 2 & 0\\
0 & 0 & 0 & 0 & 0 & 0 & 0\\
\end{tabular}
\end{center}

\subsection*{Setting Axis Titles}
\label{sec:org3309eec}
The question of the proper syntax for setting axis labels via org-plot
has occurred on the mailing list.\footnote{\url{http://www.mail-archive.com/emacs-orgmode@gnu.org/msg08669.html}} The answer is to use this:
\begin{verbatim}
#+PLOT: set:"xlabel 'Name'" set:"ylabel 'Name'"
\end{verbatim}

\section*{Reference}
\label{sec:org17dfd9d}

\subsection*{Plotting Options}
\label{sec:org3a95429}
\#\label{org7dee608}

Gnuplot options (see \href{http://gnuplot.sourceforge.net/documentation.html}{the gnuplot documentation}) accessible through
`org-plot', common gnuplot options are specifically supported, while
all other options are accessible through specification of generic set
commands, script lines, or specification of custom script files.
Possible options are\ldots{}

\begin{description}
\item[{set}] specify any gnuplot option to be set when graphing
\item[{title}] specify the title of the plot
\item[{ind}] specify which column of the table to use as the x axis
\item[{deps}] specify the columns to graph as a lisp style list,
surrounded by parenthesis and separated by spaces for
example \texttt{dep:(3 4)} to graph the third and fourth columns
(defaults to graphing all other columns aside from the ind
column).
\item[{type}] specify whether the plot will be '2d' '3d or 'grid'
\item[{with}] specify a with option to be inserted for every col being
plotted (e.g. lines, points, boxes, impulses, etc\ldots{})
defaults to 'lines'
\item[{file}] if you want to plot to a file specify the path to the
desired output file
\item[{labels}] list of labels to be used for the deps (defaults to column
headers if they exist)
\item[{line}] specify an entire line to be inserted in the gnuplot script
\item[{map}] when plotting 3d or grid types, set this to true to graph a
flat mapping rather than a 3d slope
\item[{script}] if you want total control you can specify a script file
(place the file name inside quotes) which will be used to
plot, before plotting every instance of \$datafile in the
specified script will be replaced with the path to the
generated data file.  Note even if you set this option you
may still want to specify the plot type, as that can
impact the content of the data file.
\item[{timefmt}] if there is time and/or date data to be plotted, set the
format.  For example, \texttt{timefmt:\%Y-\%m-\%d} if the data look
like \texttt{2008-03-25}.
\end{description}
\end{document}